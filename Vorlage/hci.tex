\documentclass{beamer}
\setbeamercovered{dynamic}

\usepackage[ngerman]{babel}
\usepackage{graphicx}
\usepackage[utf8]{inputenc}     % deutsche Sonderzeichen
\usepackage[T1]{fontenc}
\usepackage{url}
\usepackage{amssymb, amsmath}   % Pakete für Mathe-Umgebungen und -Symbole
\usepackage{hyperref}
\usepackage{csquotes}

\usepackage[style=apa,sortcites=true,sorting=nyt,backend=biber]{biblatex}
\addbibresource{resources/bib.bib}
\DeclareLanguageMapping{german}{german-apa}

% Zwei Versionen des Titles - s.g. short title für footer und der normale Titel für die Titelseite
\title[Usability Testing \& UX]{Usability, Usability Testing \& User Experience}
\author{Lukas Jox \textbar\ Miriam Nick}
\date{08. Januar 2019}
\institute
{Seminar: Mensch-Computer-Interaktion\\[0.7em] Institut für Psychologie und Pädagogik}

% Wenn dir die Präsentation nicht gefällt: Man kann das Aussehen dramatisch verändert, indem man einfach hier Thema und Farbschema ändert - im Internet findet man etliche Übersichten, wie Alternativen aussehen
% Ich passe mich da gerne völlig deinen Wünschen an, Design ist nicht so mein Ding... ;-)
\usetheme{Warsaw}
\usecolortheme{beaver}

\beamertemplatenavigationsymbolsempty
\setbeamercovered{transparent}


\begin{document}

% Der Hauptunterschied zu normalen LaTeX-Dokumenten liegt darin, dass man jedes "Frame" einzeln definiert - auf denen kann man dann mit den vertrauten Instrumenten und einigen anderen Arbeiten.
\begin{frame}
\titlepage
\end{frame}

\begin{frame}
\tableofcontents
\end{frame}


%showing page number
\expandafter\def\expandafter\insertshorttitle\expandafter{%
  \insertshorttitle\hfill%
  \insertframenumber$\vert$\inserttotalframenumber}
  
% Auch sections usw. können wie gewohnt verwendet werdet - allerdings werden diese nicht selbstständig angezeigt, stattdessen kann man einzelnen Frames einen \frametitle verpassen
\section{Einführung in Beamer}
\subsection{Frames}

\begin{frame}
\frametitle{Beispielfolie}
Folien können mit Bekannten Elementen oder einfach mit Text gefüllt werden:
\vspace{0,5cm}
\begin{itemize}
\item Etwa Aufzählungen
\item Figures (für Beispiele, siehe das vollständige Dokument im Ordner Beispiel)
\item Boxen (s. nächste Folie)
\item Und wahrscheinlich noch manches mehr
\end{itemize}
\end{frame}

\begin{frame}
\frametitle{Beispiele für Blocks}

\only<1-3>{\begin{block}{}
Ein einfacher, leerer Block
\end{block}}

\begin{block}{Ein Block mit Titel und einer numerischen Aufzählung}
\begin{enumerate}
\item Da \LaTeX\ nur PDF-Dokumente erstellt, müssen wir natürlich ohne Animationen usw. auskommen.
\pause
\item Zu diesem Zweck werden einfach mehrere Versionen der Folie mit Teilen des Gesamt-Inhalts erstellt.
\pause
\item Dazu dient der \char`\\pause-Befehl. Für komplexere Kombinationen kann man auch Elemente mit \char`\\only$<$$x$-$y$$>$\{\} umgeben, um es nur für die Folien-Versionen $x$ bis $y$ anzuzeigen, wie beim folgenden Stichpunkt und dem leeren Block:
\only<4>{\item Dass die weiteren Items schon transparent angezeigt werden liegt an der \char`\\setbeamercovered\{transparent\}-Option in der Präambel, diese kann auch einfach rausgenommen (oder der Grad der Transparenz angepasst) werden.}
\end{enumerate}
\end{block}
\end{frame}

\begin{frame}
Natürlich war die vorangehende Folie mit dem ganzen Inhalt völlig überfüllt, das sollte in echten Präsentationen nicht so aussehen ...
\end{frame}

\subsection{Zitationen}

\begin{frame}
\frametitle{Beispielfolie mit Zitation von \textcite{Frankl1996})}
Falls du noch nicht mit APA-cite gearbeitet hast: Wir haben zwei Haupt-kommandos:\\
\ \\
\texttt{\char`\\textcite\{\}} um so \textcite{Frankl1996} im Text zu zitieren und \char`\\parencite\{\} um Zitationen in Klammern anzugeben \parencite{Frankl1996}.\\
Eine weitere Alternative, etwa um mehrere Zitationen APA-gerecht in einer Klammer darzustellen, sind \char`\\citeauthor\{\} und \char`\\citeyear\{\} (z.B., \citeauthor{Frankl1996}, \citeyear{Frankl1996}; \citeauthor{Frankl2015}, \citeyear{Frankl2015}). 
\end{frame}

\begin{frame}
\printbibliography
\end{frame}

\end{document}